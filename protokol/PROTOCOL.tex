\documentclass{article}
\linespread{1.15}
\usepackage[utf8]{inputenc}
\usepackage{geometry}
\usepackage{graphicx}
\usepackage{polski}
\usepackage{subcaption}
\usepackage{indentfirst}
\usepackage{amsmath}
\usepackage{hyperref}
\usepackage{listings}

\begin{document}
	\title{IPC Chat Service — projekt zaliczeniowy}
	\date{Politechnika Poznańska\\ styczeń 2023}
	\author{Igor Szczepaniak \\ inf151918}
	\maketitle
	\section{Struktura wiadomości kolejki}
	\begin{lstlisting}[language=C]
		struct{
			long mtype;
			int code;
			char sender[];
			char receiver[];
			char mtext[];
		}msgbuff;
	\end{lstlisting}

	\begin{itemize}
		\item \verb+mtype+ - typ wysyłanej wiadomości
		\item \verb+code+ - kod informujący o błędzie
		\item \verb+sender+ - nadawca wiadomości
		\item \verb+receiver+ - odbiorca wiadomości
		\item \verb+mtext+ - treść wiadomości
	\end{itemize}
	
	\newpage

	\section{Opis typów wiadomości}
		Serwer nasłuchuje wszystkich wiadomości o typach podanych poniżej. Użytkownicy nasłuchują wiadomości jedynie o unikalnym typie, przydzielanym indywidualnie przez serwer.
	\begin{itemize}
		\item 1 - żądanie logowania
		\item 2 - żądanie listy zalogowanych użytkowników
		\item 3 - żądanie wylogowania
		\item 4 - żadanie wysłania wiadomości do użytkownika
		\item 5 - żądanie listy grup użytkowników
		\item 6 - żądanie wysłania wiadomości do grupy użytkowników
		\item 7 - żądanie dołączenia do grupy
		\item 8 - żądanie opuszczenia grupy
		\item $\left[MAX\_USERS; \; 2\times MAX\_USERS\right]$ - typy wiadomości, na które oczekują kolejni użytkownicy
	\end{itemize}

\end{document}
